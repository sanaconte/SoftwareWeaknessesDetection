\abstractEN % Do NOT modify this line

Software industry plays an essential role in modern world in almost all fields. Vulnerabilities are predominant in software systems and can result in a negative impact to the computer security. Although there are tools to detect vulnerable code, their accuracy and efficacy is still a challenging research question. To define features that identify vulnerabilities, many existing solutions require hard work from human experts. The constant increasing number of revealed security vulnerabilities have become an important concern in the software industry and in the field of cybersecurity, implying that the current approaches for vulnerability detection demand further improvement. This has motivated researchers in the software engineering and cybersecurity communities to apply machine learning for patterns recognition and characteristics of vulnerable code. Following this research line, this work presents a machine learning based vulnerability detection system that uses static-code analysis to extract dependencies in the code and build data features from these. The dataset was collected from the National Vulnerability Database (NVD) and test cases NIST SAMATE project and contains Java code as selected target programming language with Null pointer deference and command injections vulnerabilities as selected weaknesses. The data samples were generated from the source code of the vulnerable files by utilizing a control flow graph (CFG) to extract features. Data-flow analysis techniques were also used for feature extraction. Experimental results demonstrate that our tool can achieve significantly fewer false negatives (with a reasonable number of false positives) compared to other approaches. We further applied the tool to real software products and were able to identify vulnerabilities, despite the number of false positives.

% Keywords of abstract in English
\begin{keywords}
control flow graph, Data-flow analysis, feature extraction, machine learning, reaching definitions, static-code analysis, vulnerability detection.
\end{keywords} 
